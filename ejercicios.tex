% Options for packages loaded elsewhere
\PassOptionsToPackage{unicode}{hyperref}
\PassOptionsToPackage{hyphens}{url}
%
\documentclass[
]{article}
\usepackage{amsmath,amssymb}
\usepackage{iftex}
\ifPDFTeX
  \usepackage[T1]{fontenc}
  \usepackage[utf8]{inputenc}
  \usepackage{textcomp} % provide euro and other symbols
\else % if luatex or xetex
  \usepackage{unicode-math} % this also loads fontspec
  \defaultfontfeatures{Scale=MatchLowercase}
  \defaultfontfeatures[\rmfamily]{Ligatures=TeX,Scale=1}
\fi
\usepackage{lmodern}
\ifPDFTeX\else
  % xetex/luatex font selection
\fi
% Use upquote if available, for straight quotes in verbatim environments
\IfFileExists{upquote.sty}{\usepackage{upquote}}{}
\IfFileExists{microtype.sty}{% use microtype if available
  \usepackage[]{microtype}
  \UseMicrotypeSet[protrusion]{basicmath} % disable protrusion for tt fonts
}{}
\makeatletter
\@ifundefined{KOMAClassName}{% if non-KOMA class
  \IfFileExists{parskip.sty}{%
    \usepackage{parskip}
  }{% else
    \setlength{\parindent}{0pt}
    \setlength{\parskip}{6pt plus 2pt minus 1pt}}
}{% if KOMA class
  \KOMAoptions{parskip=half}}
\makeatother
\usepackage{xcolor}
\usepackage[margin=1in]{geometry}
\usepackage{color}
\usepackage{fancyvrb}
\newcommand{\VerbBar}{|}
\newcommand{\VERB}{\Verb[commandchars=\\\{\}]}
\DefineVerbatimEnvironment{Highlighting}{Verbatim}{commandchars=\\\{\}}
% Add ',fontsize=\small' for more characters per line
\usepackage{framed}
\definecolor{shadecolor}{RGB}{248,248,248}
\newenvironment{Shaded}{\begin{snugshade}}{\end{snugshade}}
\newcommand{\AlertTok}[1]{\textcolor[rgb]{0.94,0.16,0.16}{#1}}
\newcommand{\AnnotationTok}[1]{\textcolor[rgb]{0.56,0.35,0.01}{\textbf{\textit{#1}}}}
\newcommand{\AttributeTok}[1]{\textcolor[rgb]{0.13,0.29,0.53}{#1}}
\newcommand{\BaseNTok}[1]{\textcolor[rgb]{0.00,0.00,0.81}{#1}}
\newcommand{\BuiltInTok}[1]{#1}
\newcommand{\CharTok}[1]{\textcolor[rgb]{0.31,0.60,0.02}{#1}}
\newcommand{\CommentTok}[1]{\textcolor[rgb]{0.56,0.35,0.01}{\textit{#1}}}
\newcommand{\CommentVarTok}[1]{\textcolor[rgb]{0.56,0.35,0.01}{\textbf{\textit{#1}}}}
\newcommand{\ConstantTok}[1]{\textcolor[rgb]{0.56,0.35,0.01}{#1}}
\newcommand{\ControlFlowTok}[1]{\textcolor[rgb]{0.13,0.29,0.53}{\textbf{#1}}}
\newcommand{\DataTypeTok}[1]{\textcolor[rgb]{0.13,0.29,0.53}{#1}}
\newcommand{\DecValTok}[1]{\textcolor[rgb]{0.00,0.00,0.81}{#1}}
\newcommand{\DocumentationTok}[1]{\textcolor[rgb]{0.56,0.35,0.01}{\textbf{\textit{#1}}}}
\newcommand{\ErrorTok}[1]{\textcolor[rgb]{0.64,0.00,0.00}{\textbf{#1}}}
\newcommand{\ExtensionTok}[1]{#1}
\newcommand{\FloatTok}[1]{\textcolor[rgb]{0.00,0.00,0.81}{#1}}
\newcommand{\FunctionTok}[1]{\textcolor[rgb]{0.13,0.29,0.53}{\textbf{#1}}}
\newcommand{\ImportTok}[1]{#1}
\newcommand{\InformationTok}[1]{\textcolor[rgb]{0.56,0.35,0.01}{\textbf{\textit{#1}}}}
\newcommand{\KeywordTok}[1]{\textcolor[rgb]{0.13,0.29,0.53}{\textbf{#1}}}
\newcommand{\NormalTok}[1]{#1}
\newcommand{\OperatorTok}[1]{\textcolor[rgb]{0.81,0.36,0.00}{\textbf{#1}}}
\newcommand{\OtherTok}[1]{\textcolor[rgb]{0.56,0.35,0.01}{#1}}
\newcommand{\PreprocessorTok}[1]{\textcolor[rgb]{0.56,0.35,0.01}{\textit{#1}}}
\newcommand{\RegionMarkerTok}[1]{#1}
\newcommand{\SpecialCharTok}[1]{\textcolor[rgb]{0.81,0.36,0.00}{\textbf{#1}}}
\newcommand{\SpecialStringTok}[1]{\textcolor[rgb]{0.31,0.60,0.02}{#1}}
\newcommand{\StringTok}[1]{\textcolor[rgb]{0.31,0.60,0.02}{#1}}
\newcommand{\VariableTok}[1]{\textcolor[rgb]{0.00,0.00,0.00}{#1}}
\newcommand{\VerbatimStringTok}[1]{\textcolor[rgb]{0.31,0.60,0.02}{#1}}
\newcommand{\WarningTok}[1]{\textcolor[rgb]{0.56,0.35,0.01}{\textbf{\textit{#1}}}}
\usepackage{graphicx}
\makeatletter
\def\maxwidth{\ifdim\Gin@nat@width>\linewidth\linewidth\else\Gin@nat@width\fi}
\def\maxheight{\ifdim\Gin@nat@height>\textheight\textheight\else\Gin@nat@height\fi}
\makeatother
% Scale images if necessary, so that they will not overflow the page
% margins by default, and it is still possible to overwrite the defaults
% using explicit options in \includegraphics[width, height, ...]{}
\setkeys{Gin}{width=\maxwidth,height=\maxheight,keepaspectratio}
% Set default figure placement to htbp
\makeatletter
\def\fps@figure{htbp}
\makeatother
\setlength{\emergencystretch}{3em} % prevent overfull lines
\providecommand{\tightlist}{%
  \setlength{\itemsep}{0pt}\setlength{\parskip}{0pt}}
\setcounter{secnumdepth}{-\maxdimen} % remove section numbering
\ifLuaTeX
  \usepackage{selnolig}  % disable illegal ligatures
\fi
\IfFileExists{bookmark.sty}{\usepackage{bookmark}}{\usepackage{hyperref}}
\IfFileExists{xurl.sty}{\usepackage{xurl}}{} % add URL line breaks if available
\urlstyle{same}
\hypersetup{
  pdftitle={Resolución ejercicios semana 2},
  hidelinks,
  pdfcreator={LaTeX via pandoc}}

\title{Resolución ejercicios semana 2}
\author{}
\date{\vspace{-2.5em}}

\begin{document}
\maketitle

\hypertarget{dado-cargado}{%
\section{Dado cargado}\label{dado-cargado}}

\begin{Shaded}
\begin{Highlighting}[]
\NormalTok{lanzamiento\_dado }\OtherTok{\textless{}{-}} \FunctionTok{sample}\NormalTok{(}\DecValTok{1}\SpecialCharTok{:}\DecValTok{6}\NormalTok{, }\AttributeTok{size =} \DecValTok{1}\NormalTok{)}
\NormalTok{lanzamiento\_moneda }\OtherTok{\textless{}{-}} \FunctionTok{runif}\NormalTok{(}\AttributeTok{n =} \DecValTok{1}\NormalTok{)}

\FunctionTok{print}\NormalTok{(lanzamiento\_dado)}
\end{Highlighting}
\end{Shaded}

\begin{verbatim}
## [1] 2
\end{verbatim}

\begin{Shaded}
\begin{Highlighting}[]
\FunctionTok{print}\NormalTok{(lanzamiento\_moneda)}
\end{Highlighting}
\end{Shaded}

\begin{verbatim}
## [1] 0.865746
\end{verbatim}

\begin{Shaded}
\begin{Highlighting}[]
\ControlFlowTok{if}\NormalTok{ (lanzamiento\_moneda }\SpecialCharTok{\textgreater{}} \FloatTok{0.6} \SpecialCharTok{\&}\NormalTok{ lanzamiento\_dado }\SpecialCharTok{\%in\%} \FunctionTok{c}\NormalTok{(}\DecValTok{1}\NormalTok{, }\DecValTok{3}\NormalTok{, }\DecValTok{5}\NormalTok{)  ) \{}
  \FunctionTok{print}\NormalTok{(}\StringTok{"¡Premio para el curso!"}\NormalTok{)  }
  
\NormalTok{\} }\ControlFlowTok{else}\NormalTok{ \{}
  \FunctionTok{print}\NormalTok{(}\StringTok{"Pucha, sigue participando"}\NormalTok{)}
\NormalTok{\}}
\end{Highlighting}
\end{Shaded}

\begin{verbatim}
## [1] "Pucha, sigue participando"
\end{verbatim}

\hypertarget{robot-reponedor}{%
\section{Robot reponedor}\label{robot-reponedor}}

Tienes que construir el flujo de trabajo de un robot que repone
mercadería en un supermercado. Las acciones del robot son las
siguientes:

\begin{itemize}
\tightlist
\item
  Si la góndola está llena, se debe imprimir ``seguir a la siguiente
  góndola''
\item
  Si la góndola tiene 80\% o más de capacidad, se debe imprimir ``nivel
  aceptable. Volver en 4 horas''
\item
  Si la góndola está entre 50\% (incluyendo) y 80\%, se debe imprimir:
  ``nivel medio. Volver en 2 horas''
\item
  Si la góndola tiene menos de 50\%, se debe imprimir ``nivel crítico.
  Avisar al supervisor''
\end{itemize}

La variable de stock, se inicializa con el valor 51

\begin{Shaded}
\begin{Highlighting}[]
\NormalTok{stock }\OtherTok{\textless{}{-}} \DecValTok{100}
\end{Highlighting}
\end{Shaded}

\begin{Shaded}
\begin{Highlighting}[]
\ControlFlowTok{if}\NormalTok{ (stock }\SpecialCharTok{==} \DecValTok{100}\NormalTok{) \{}
  \FunctionTok{print}\NormalTok{(}\StringTok{"seguir a la siguiente góndola"}\NormalTok{)}
\NormalTok{\} }\ControlFlowTok{else} \ControlFlowTok{if}\NormalTok{ (stock }\SpecialCharTok{\textgreater{}=} \DecValTok{80}\NormalTok{) \{}
  \FunctionTok{print}\NormalTok{(}\StringTok{"nivel aceptable. Volver en 4 horas"}\NormalTok{)}
\NormalTok{\} }\ControlFlowTok{else} \ControlFlowTok{if}\NormalTok{ (stock }\SpecialCharTok{\textgreater{}=} \DecValTok{50}\NormalTok{) \{}
  \FunctionTok{print}\NormalTok{(}\StringTok{"nivel medio. Volver en 2 horas"}\NormalTok{)}
\NormalTok{\} }\ControlFlowTok{else}\NormalTok{ \{}
  \FunctionTok{print}\NormalTok{(}\StringTok{"nivel crítico. Avisar al supervidor"}\NormalTok{)}
\NormalTok{\}}
\end{Highlighting}
\end{Shaded}

\begin{verbatim}
## [1] "seguir a la siguiente góndola"
\end{verbatim}

\hypertarget{subsidio-desempleo-joven}{%
\section{Subsidio desempleo joven}\label{subsidio-desempleo-joven}}

Usted trabaja en una municipalidad y le encargan crear un programa que
asigne un subsidio a las personas, en función de su edad y situación en
el empleo.

Las reglas son las siguientes: - Edad \textless{} 25 y desocupado
==\textgreater{} 125.000 - Edad \textgreater= 25 y desocupado
==\textgreater{} 200.000 - Ocupado ==\textgreater{} 0

Los estados de ocupación pueden ser: ocupado o desocupado

En caso de que se cumpla una condición, el programa debe responder lo
siguiente: ``usted tiene un subsidio de (monto recibido)''

Considere que el programa cuenta con un fondo inicial de 600.000, que va
disminuyendo conforme se asignan los recursos. En caso de que el fondo
se agote, el programa debe enviar el siguiente mensaje: ``ya no quedan
recursos''. En caso de que el saldo no alcance a cubrir el subsidio, el
programa debe entregar lo que tenga disponible y dejar el fondo en 0.

\begin{Shaded}
\begin{Highlighting}[]
\CommentTok{\# Parámetros para probar}
\NormalTok{fondo }\OtherTok{\textless{}{-}} \DecValTok{600000}
\NormalTok{edad }\OtherTok{\textless{}{-}} \DecValTok{27}
\NormalTok{situacion\_empleo }\OtherTok{\textless{}{-}} \StringTok{"desocupado"} 


\CommentTok{\# Si no quedan fondos, el programa no sigue}
\ControlFlowTok{if}\NormalTok{ (fondo }\SpecialCharTok{\textgreater{}} \DecValTok{0}\NormalTok{) \{}
  
  \CommentTok{\# Subsidio alto}
  \ControlFlowTok{if}\NormalTok{ (edad  }\SpecialCharTok{\textless{}} \DecValTok{25} \SpecialCharTok{\&}\NormalTok{ situacion\_empleo }\SpecialCharTok{==} \StringTok{"desocupado"}\NormalTok{ ) \{}
    
\NormalTok{    subsidio }\OtherTok{\textless{}{-}} \DecValTok{125000}
\NormalTok{    diferencia }\OtherTok{\textless{}{-}}\NormalTok{ fondo }\SpecialCharTok{{-}}\NormalTok{  subsidio}
    \ControlFlowTok{if}\NormalTok{ (diferencia }\SpecialCharTok{\textgreater{}} \DecValTok{0}\NormalTok{ ) \{}
      \FunctionTok{print}\NormalTok{(}\FunctionTok{paste0}\NormalTok{(}\StringTok{"usted tiene un subsidio de "}\NormalTok{, subsidio))}
\NormalTok{    \} }\ControlFlowTok{else}\NormalTok{ \{}
      \FunctionTok{print}\NormalTok{(}\FunctionTok{paste0}\NormalTok{(}\StringTok{"usted tiene un subsidio de "}\NormalTok{, fondo ))}
\NormalTok{    \}}
    
  \CommentTok{\# Subsidio bajo}
\NormalTok{  \} }\ControlFlowTok{else} \ControlFlowTok{if}\NormalTok{ (edad  }\SpecialCharTok{\textgreater{}=} \DecValTok{25} \SpecialCharTok{\&}\NormalTok{ situacion\_empleo }\SpecialCharTok{==} \StringTok{"desocupado"}\NormalTok{) \{}
\NormalTok{    subsidio }\OtherTok{\textless{}{-}} \DecValTok{200000}
\NormalTok{    diferencia }\OtherTok{\textless{}{-}}\NormalTok{ fondo }\SpecialCharTok{{-}}\NormalTok{  subsidio}

    \ControlFlowTok{if}\NormalTok{ (diferencia }\SpecialCharTok{\textgreater{}} \DecValTok{0}\NormalTok{ ) \{}
      \FunctionTok{print}\NormalTok{(}\FunctionTok{paste0}\NormalTok{(}\StringTok{"usted tiene un subsidio de "}\NormalTok{, subsidio))}
\NormalTok{    \} }\ControlFlowTok{else}\NormalTok{ \{}
      \FunctionTok{print}\NormalTok{(}\FunctionTok{paste0}\NormalTok{(}\StringTok{"usted tiene un subsidio de "}\NormalTok{, fondo ))}
\NormalTok{    \}}
  \CommentTok{\# Sin subsidio}
\NormalTok{  \} }\ControlFlowTok{else}\NormalTok{ \{}
\NormalTok{    subsidio }\OtherTok{\textless{}{-}} \DecValTok{0}
    \FunctionTok{print}\NormalTok{(}\FunctionTok{paste0}\NormalTok{(}\StringTok{"usted tiene un subsidio de "}\NormalTok{, subsidio))}
\NormalTok{  \}}
  \CommentTok{\# Actualizar fondo}
\NormalTok{  fondo }\OtherTok{\textless{}{-}}\NormalTok{ fondo }\SpecialCharTok{{-}}\NormalTok{ subsidio}
  \FunctionTok{print}\NormalTok{(}\FunctionTok{paste}\NormalTok{(}\StringTok{"El saldo es"}\NormalTok{, fondo ))}
\NormalTok{\} }\ControlFlowTok{else}\NormalTok{ \{}
  \FunctionTok{print}\NormalTok{(}\StringTok{"ya no quedan recursos"}\NormalTok{)}
\NormalTok{\}}
\end{Highlighting}
\end{Shaded}

\begin{verbatim}
## [1] "usted tiene un subsidio de 2e+05"
## [1] "El saldo es 4e+05"
\end{verbatim}

\hypertarget{lanzamientos-de-moneda}{%
\subsection{10.000 lanzamientos de
moneda}\label{lanzamientos-de-moneda}}

\begin{Shaded}
\begin{Highlighting}[]
\NormalTok{cara }\OtherTok{\textless{}{-}} \DecValTok{0}
\NormalTok{sello }\OtherTok{=} \DecValTok{0}
\NormalTok{experimentos }\OtherTok{\textless{}{-}} \DecValTok{10000} 
\ControlFlowTok{for}\NormalTok{ (i }\ControlFlowTok{in} \DecValTok{1}\SpecialCharTok{:}\NormalTok{experimentos) \{}
\NormalTok{  lanzamiento }\OtherTok{\textless{}{-}} \FunctionTok{runif}\NormalTok{(}\AttributeTok{n =} \DecValTok{1}\NormalTok{)}
  \ControlFlowTok{if}\NormalTok{ (lanzamiento }\SpecialCharTok{\textless{}=} \FloatTok{0.5}\NormalTok{) \{}
\NormalTok{    cara }\OtherTok{\textless{}{-}}\NormalTok{ cara }\SpecialCharTok{+} \DecValTok{1}
\NormalTok{  \} }\ControlFlowTok{else}\NormalTok{ \{}
\NormalTok{    sello }\OtherTok{=}\NormalTok{ sello }\SpecialCharTok{+} \DecValTok{1}
\NormalTok{  \}}
\NormalTok{\}}
\NormalTok{cara }\SpecialCharTok{/}\NormalTok{ experimentos}
\end{Highlighting}
\end{Shaded}

\begin{verbatim}
## [1] 0.5079
\end{verbatim}

\end{document}
